\documentclass[11pt,letterpaper]{article}
\usepackage{amsmath, amssymb, amsbsy}
\usepackage{cite, graphicx}
\usepackage{geometry}
\usepackage{subcaption}

\usepackage[usenames,dvipsnames]{xcolor} 

\usepackage[utf8]{inputenc}

\geometry{letterpaper,nohead,margin=1.4in}
\parindent1em
\parskip0pc
\linespread{1.0}
\pagestyle{plain}

\newcommand{\com}[1]{\hspace{2em}\textrm{#1}} % comments
\newcommand{\sinc}[0]{\textrm{sinc}}
\newcommand{\sign}[0]{\textrm{sign}}
\newcommand{\vv}[1]{\mathbf{#1}} % vector
\newcommand{\p}[0]{\mathbb{P}}
\newcommand{\e}[0]{\mathbb{E}}
\newcommand{\var}[0]{\textrm{Var}}

\title{Learning From Data Problems: Chapter II}
\date{}
\author{J. David Giese}

\begin{document}
\maketitle

\section*{Exercise 2.1}
The breaking point for (1) is $N = 2$ because $(1, -1) \not\in \mathcal{H}(\vv{x}_1, \vv{x}_2)$.
\\\\
The breaking point for (2) is $N = 3$ because $(1, -1, 1) \not\in \mathcal{H}(\vv{x}_1, \vv{x}_2, \vv{x}_3)$.
\\\\
There is no breaking point for (3) because every dichotomy can be generated by $\mathcal{H}$.

\section*{Exercise 2.2}

a)
\\\\
For (1), we have $m_\mathcal{H}(N) \le \binom{N}{1} + \binom{N}{0} = N + 1$, which is true.
\\\\
For (2), we have $m_\mathcal{H}(N) \le \binom{N}{2} + \binom{N}{1} + \binom{N}{0} = N^2/2 + N/2 + 1$, which is true.
\\\\
There is no bound for (3) as there is no break point.
\\\\
b)
\\\\
No, because if $m_\mathcal{H}(N) < 2^N$ then there must be a break point, however if there is a break point it will be polynomial bounded.

\section*{Exercise 2.3}
The Vapnik-Chervonenkis dimension is 1, 2, and $\infty$ respectively.

\end{document}

